\documentclass{article}

\input{preamble-articles}

%%%%%%%%%%%%%%%%%%%%%%%%%%%%%%%%%%%%%%%%%%%%%%%%%%%%%%%%%%%%%%%%%%%%%%%%%%%%%%%%
%%%% We define a command \@ifnextcharamong accepting an arbitrary number of
%%%% arguments. The first is what it should do if a match is found, the second
%%%% contains what it should do when no match is found; all the other arguments
%%%% are the things it tries to find as the next character.
%%%%
%%%% For example \@ifnextcharamong{#1}{#2}{*}{\bgroup} expands #1 if the next
%%%% character is a * or a \bgroup and it expands #2 otherwise.

\newcommand{\projective}[1]{\mathbb{R}\mathsf{P}^{#1}}
\newcommand{\antipodal}[1]{\mathsf{antipodal}_{#1}}

\makeatletter
\newcommand{\@ifnextcharamong}[2]
  {\@ifnextchar\bgroup{\@@ifnextchar{#1}{\@@ifnextcharamong{#1}{#2}}}{#2}}
\newcommand{\@@ifnextchar}[3]{\@ifnextchar{#3}{#1}{#2}}
\newcommand{\@@ifnextcharamong}[3]{\@ifnextcharamong{#1}{#2}}
\makeatother

\newcommand{\ucomp}[1]{\hat{#1}}
\newcommand{\finset}[1]{{[#1]}}

\makeatletter
\newcommand{\higherequifibsf}{\mathcal}
\newcommand{\higherequifib}[2]{\higherequifibsf{#1}(#2)}
\newcommand{\underlyinggraph}[1]{U(#1)}
\newcommand{\theequifib}[1]{{\def\higherequifibsf{}#1}}
\makeatother

\newcommand{\loopspace}[2][]{\typefont{\Omega}^{#1}(#2)}
\newcommand{\join}[2]{{#1}*{#2}}

%%%%%%%%%%%%%%%%%%%%%%%%%%%%%%%%%%%%%%%%%%%%%%%%%%%%%%%%%%%%%%%%%%%%%%%%%%%%%%%%
\title{K-theory}
\date{\today}
\author{Ulrik Buchholtz \& Egbert Rijke}

\begin{document}

\maketitle

\tableofcontents

\section{The group structures on $\Sn^1$, $\Sn^3$ and $\Sn^7$}
Classically, the $1$, $3$ and $7$ dimensional spheres are subspaces of $\mathbb{R}^2$,
$\mathbb{R}^4$ and $\mathbb{R}^8$, respectively. Each of these vector spaces can be
given the structure of normed division algebras, and we get the complex numbers
$\mathbb{C}$, Hamilton's quaternions $\mathbb{H}$, and the octonions $\mathbb{O}$
of Graves and Cayley. Since, in each of these algebras, the product preserves
the norm, the unit sphere is a subgroup of the multiplicative group.

In the case of the circle, this group structure has already been formalized and
appears in \cite{TheBook}. Formalizing the notion of group structure on a type
of truncation level strictly bigger than $1$, is tricky, because it involves
higher coherences. However, a class of types with a natural group structure
consists of types of the form $\Omega(X)$, and types which have group structure
judgmentally are types of the form $\mathsf{Aut}(X)$. 

Classically, it is known that $B(\Sn^3)=\mathbb{H}\mathsf{P}^\infty=K(\mathbb{Q},4)$. 

\begin{defn}
A group structure on a type $X$ consists of a connected pointed type $X'$, such
that $X$ is the loop space of $X'$.
\end{defn}

\subsection{The group structure of $\Sn^0$}

Of course, $\Sn^0$ is just the type $\bool$, so it has the group structure of
$\Z_2$.

\begin{defn}
We define multiplication $\mu_0:\Sn^0\times\Sn^0\to\Sn^0$ by
\begin{align*}
\mu_0(\north,\north) & \defeq \north \\
\mu_0(\north,\south) & \defeq \south \\
\mu_0(\south,\north) & \defeq \south \\
\mu_0(\south,\south) & \defeq \north
\end{align*}
\end{defn}

\subsection{The group structure of $\Sn^1$}
This has already been described in \cite{TheBook}. We do it slightly differently,
to be able to generalize it to dimensions 3 and 7.

\begin{defn}
The circle $\Sn^1$ is the join $\join{\Sn^0}{\Sn^0}$. 
We denote the two point constructors $\Sn^0\to\Sn^1$ by $e_0$ and $e_1$. 
\end{defn}

The circle looks like this
\begin{equation*}
\begin{tikzcd}[column sep=small]
& e_1(\north) \arrow[dl,bend right=35,equals,swap,"\alpha_2"] \\
e_0(\south) \arrow[dr,bend right=35,equals,swap,"\alpha_3"] & & e_0(\north) \arrow[ul,bend right=35,equals,swap,"\alpha_1"] \\
& e_1(\south) \arrow[ur,bend right=35,equals,swap,"\alpha_4"] 
\end{tikzcd}
\end{equation*}
Just as with the circle in the complex plane, the group operation on the circle
will be given by rotation. So, $e_0(\north)$ will be the unit of the group,
$e_1(\north)$ will rotate a quarter counter-clockwise, $e_0(\south)$ will rotate
half around counter-clockwise, and $e_1(\south)$ will rotate three quarters
counter-clockwise.

\begin{defn}
The circle has two $\Sn^0$-actions, i.e. functions $a_0$ and $a_1$ of type
\begin{equation*}
\Sn^0\times\Sn^1\to\Sn^1.
\end{equation*}
\end{defn}

\begin{proof}[Construction]
Note that $\Sn^0\times\Sn^1$ is a pushout, so we may define the $\Sn^0$-action $a_0$
on $\Sn^1$ by
\begin{equation*}
\begin{tikzcd}
\Sn^0\times\Sn^0\times\Sn^0 \arrow[r,"\pi_{1,3}"] \arrow[d,swap,"\pi_{1,2}"] & \Sn^0\times\Sn^0 \arrow[d,"\idfunc\times e_1"] \arrow[r,"\mu_0"] & \Sn^0 \arrow[dd,"e_1"] \\
\Sn^0\times\Sn^0 \arrow[d,swap,"\mu_0"] \arrow[r,"\idfunc\times e_0"] & \Sn^0\times\Sn^1 \arrow[dr,densely dotted,"a_0"] \\
\Sn^0 \arrow[rr,swap,"e_0"] & & \Sn^1
\end{tikzcd}
\end{equation*}
Thus, we have to show that the square commutes. This, we do via the table
\begin{center}
\begin{tabular}{ccc|cc|l}
$x$ & $y$ & $z$ & $e_0(\mu_0(x,y))$ & $e_1(\mu_0(x,z))$ & identification \\
\midrule
$\north$ & $\north$ & $\north$ & $e_0(\north)$ & $e_1(\north)$ & $\alpha_1$ \\
$\north$ & $\north$ & $\south$ & $e_0(\north)$ & $e_1(\south)$ & $\ct{\alpha_1}{\alpha_2}{\alpha_3}$ \\
$\north$ & $\south$ & $\north$ & $e_0(\south)$ & $e_1(\north)$ & $\ct{\alpha_3}{\alpha_4}{\alpha_1}$ \\
$\north$ & $\south$ & $\south$ & $e_0(\south)$ & $e_1(\south)$ & $\alpha_3$ \\
$\south$ & $\north$ & $\north$ & $e_0(\south)$ & $e_1(\north)$ & $\ct{\alpha_3}{\alpha_4}{\alpha_1}$ \\
$\south$ & $\north$ & $\south$ & $e_0(\south)$ & $e_1(\south)$ & $\alpha_3$ \\
$\south$ & $\south$ & $\north$ & $e_0(\north)$ & $e_1(\north)$ & $\alpha_1$ \\
$\south$ & $\south$ & $\south$ & $e_0(\north)$ & $e_1(\south)$ & $\ct{\alpha_1}{\alpha_2}{\alpha_3}$
\end{tabular}
\end{center}
Similarly, we may define the $\Sn^0$-action $a_1$ on $\Sn^1$ by
\begin{equation*}
\begin{tikzcd}
\Sn^0\times\Sn^0\times\Sn^0 \arrow[r,"\pi_{1,3}"] \arrow[d,swap,"\pi_{1,2}"] & \Sn^0\times\Sn^0 \arrow[d,"\idfunc\times e_1"] \arrow[r,"\mu_0"] & \Sn^0 \arrow[dd,"e_0"] \\
\Sn^0\times\Sn^0 \arrow[d,swap,"\mu_0"] \arrow[r,"\idfunc\times e_0"] & \Sn^0\times\Sn^1 \arrow[dr,densely dotted,"a_0"] \\
\Sn^0 \arrow[rr,swap,"e_1"] & & \Sn^1
\end{tikzcd}
\end{equation*}
Thus, we have to show that the square commutes. This, we do via the table
\begin{center}
\begin{tabular}{ccc|cc|l}
$x$ & $y$ & $z$ & $e_1(\mu_0(x,y))$ & $e_0(\mu_0(x,z))$ & identification \\
\midrule
$\north$ & $\north$ & $\north$ & $e_1(\north)$ & $e_0(\north)$ & $\ct{\alpha_2}{\alpha_3}{\alpha_4}$ \\
$\north$ & $\north$ & $\south$ & $e_1(\south)$ & $e_0(\north)$ & $\alpha_4$ \\
$\north$ & $\south$ & $\north$ & $e_1(\north)$ & $e_0(\south)$ & $\alpha_2$ \\
$\north$ & $\south$ & $\south$ & $e_1(\south)$ & $e_0(\south)$ & $\ct{\alpha_4}{\alpha_1}{\alpha_2}$ \\
$\south$ & $\north$ & $\north$ & $e_1(\north)$ & $e_0(\south)$ & $\alpha_2$ \\
$\south$ & $\north$ & $\south$ & $e_1(\south)$ & $e_0(\south)$ & $\ct{\alpha_4}{\alpha_1}{\alpha_2}$ \\
$\south$ & $\south$ & $\north$ & $e_1(\north)$ & $e_0(\north)$ & $\ct{\alpha_2}{\alpha_3}{\alpha_4}$ \\
$\south$ & $\south$ & $\south$ & $e_1(\south)$ & $e_0(\north)$ & $\alpha_4$
\end{tabular}
\end{center}
\end{proof}

\begin{defn}
We define $\mu_1:\Sn^1\times\Sn^1\to\Sn^1$ to be the unique map for which the
diagram
\begin{equation*}
\begin{tikzcd}
(\Sn^1\times\Sn^1)\times\Sn^3 \arrow[r] \arrow[d] & \Sn^1\times\Sn^3 \arrow[d] \\
\Sn^1\times\Sn^3 \arrow[r] & \Sn^3
\end{tikzcd}
\end{equation*}
\end{defn}

\subsection{The group structure of $\Sn^3$}
We want to describe the group structure of $\Sn^3$ explicitly, i.e.~we want to
construct a map
\begin{equation*}
\mu_3 : \Sn^3\times \Sn^3\to\Sn^3.
\end{equation*}
Recall that $\eqv{\Sn^3}{\join{\Sn^1}{\Sn^1}}$. We write the inclusion maps
as $i,j:\Sn^1\to \join{\Sn^1}{\Sn^1}$. There is a third map worth mentioning,
corresponding to the circle in the $k$-th dimension.

\begin{defn}
We define the map $k:\Sn^1\to\join{\Sn^1}{\Sn^1}$ as the unique map for which
the diagram
\begin{equation*}
\begin{tikzcd}
\Sn^0\times\Sn^0 \arrow[r] \arrow[d] & \Sn^0 \arrow[d] \arrow[r] & \Sn^1 \arrow[dd,"i"] \\
\Sn^0 \arrow[r] \arrow[d] & \Sn^1 \arrow[dr,densely dotted,"k"] \\
\Sn^1 \arrow[rr,swap,"j"] & & \join{\Sn^1}{\Sn^1}. 
\end{tikzcd}
\end{equation*}
commutes.
\end{defn}

To define a multiplication $\mu_3:\Sn^3\times\Sn^3\to\Sn^3$, it might be easier
to first define the action of $i:\Sn^1\to\Sn^3$ on $\Sn^3$. In other words, we
want to first define a map $\mu_3(i(\blank),\blank):\Sn^1\times\Sn^3\to\Sn^3$.
Note that $\Sn^1\times\Sn^3$ is a pushout, so we may define $\mu_3(i(\blank),\blank)$
to be the unique map making the square
\begin{equation*}
\begin{tikzcd}
\Sn^1\times\Sn^1\times\Sn^1 \arrow[r] \arrow[d] & \Sn^1\times\Sn^1 \arrow[d,"\idfunc\times j"] \arrow[r,"\mu_1"] & \Sn^1 \arrow[dd,"k"] \\
\Sn^1\times\Sn^1 \arrow[r,swap,"\idfunc\times i"] \arrow[d,swap,"\mu_1"] & \Sn^1\times\Sn^3 \arrow[dr,densely dotted] \\
\Sn^1 \arrow[rr,swap,"i"] & & \Sn^3
\end{tikzcd}
\end{equation*}
commute. For this to work, we have to verify that
\begin{equation*}
\lam{x}{y}{z}i(\mu_1(x,y))=\lam{x}{y}{z}k(\mu_1(x,z))
\end{equation*}
for each $x,y,z:\Sn^1$. 


\section{Line bundles}

\begin{defn}
For any $n:\nat$, we define the \define{antipodal map} 
\begin{equation*}
\antipodal{n}:\Sn^n\to\Sn^n
\end{equation*} 
inductively, by defining $\antipodal{0}\defeq\mathsf{neg}:\bool\to\bool$, and by defining
$\antipodal{n+1}$ to be the unique map specified by
\begin{align*}
\antipodal{n+1}(\north) & \defeq\south \\
\antipodal{n+1}(\south) & \defeq\north \\
\ap{\antipodal{n+1}}{\glue(x)} & \defeq \glue(\antipodal{n}(x)).
\end{align*}
\end{defn}

\begin{defn}
We define the \define{projective space} $\projective{n}$ to be the coequalizer of the parallel pair
\begin{equation*}
\begin{tikzcd}[column sep=huge]
\Sn^n\arrow[r,yshift=.7ex,"\antipodal{n}"] \arrow[r,yshift=-.7ex,swap,"\idfunc"] & \Sn^n
\end{tikzcd}
\end{equation*}
\end{defn}

\begin{defn}
We get the following commuting diagram
\begin{equation*}
\begin{tikzcd}[column sep=huge]
\Sn^0 \arrow[r] \arrow[d,xshift=-.7ex,swap,"\mathsf{antipodal}_0"] \arrow[d,xshift=.7ex,"\idfunc"]
& \Sn^1 \arrow[r] \arrow[d,xshift=-.7ex,swap,"\mathsf{antipodal}_1"] \arrow[d,xshift=.7ex,"\idfunc"]
& \Sn^2 \arrow[r] \arrow[d,xshift=-.7ex,swap,"\mathsf{antipodal}_2"] \arrow[d,xshift=.7ex,"\idfunc"]
& \cdots \\
\Sn^0 \arrow[r] \arrow[d] & \Sn^1 \arrow[r] \arrow[d] & \Sn^2 \arrow[r] \arrow[d] & \cdots \\
\projective{0} \arrow[r,densely dotted] & \projective{1} \arrow[r,densely dotted] & \projective{2} \arrow[r,densely dotted] & \cdots
\end{tikzcd}
\end{equation*}
where the bottom maps are the uniqe maps which extend the circle inclusions.
Thus, we may define $\projective{\infty}$ as the sequential colimit of $\projective{n}$. 
\end{defn}

\begin{conj}
The type $\projective{\infty}$ is equivalent to the type $\frac{1}{2}\defeq\sm{X:\UU}\brck{X=\bool}$. 
\end{conj}

\end{document}


